\documentclass[compress]{beamer}
\usepackage{ifthen,verbatim,hyperref}

\title{iCSA08 Workflow and Performance Metrics \\ Muon-HIP Alignment}
\author{Jim Pivarski, Alexei Safonov, K\'aroly Banicz$^*$}
\institute{Texas A\&M University, $^*$FermiLab}
\date{28 March, 2008}

\newcommand{\isnote}{}
\xdefinecolor{lightyellow}{rgb}{1.,1.,0.25}
\xdefinecolor{darkblue}{rgb}{0.1,0.1,0.7}

%% Uncomment this to get annotations
%% \def\notes{\addtocounter{page}{-1}
%%            \renewcommand{\isnote}{*}
%% 	   \beamertemplateshadingbackground{lightyellow}{white}
%%            \begin{frame}
%%            \frametitle{Notes for the previous page (page \insertpagenumber)}
%%            \itemize}
%% \def\endnotes{\enditemize
%% 	      \end{frame}
%%               \beamertemplateshadingbackground{white}{white}
%%               \renewcommand{\isnote}{}}

%% Uncomment this to not get annotations
\def\notes{\comment}
\def\endnotes{\endcomment}

\setbeamertemplate{navigation symbols}{}
\setbeamertemplate{headline}{\mbox{ } \hfill
\begin{minipage}{5.5 cm}
\vspace{-0.75 cm} \small
\end{minipage} \hfill
\begin{minipage}{4.5 cm}
\vspace{-0.75 cm} \small
\begin{flushright}
\ifthenelse{\equal{\insertpagenumber}{1}}{}{Jim Pivarski \hspace{0.2 cm} \insertpagenumber\isnote/\pageref{numpages}}
\end{flushright}
\end{minipage}\mbox{\hspace{0.2 cm}}\includegraphics[height=1 cm]{../cmslogo} \hspace{0.1 cm} \includegraphics[height=1 cm]{../tamulogo} \hspace{0.01 cm} \vspace{-1.05 cm}}

\begin{document}
\frame{\titlepage}

%% \begin{notes}
%% \item This is the annotated version of my talk.
%% \item If you want the version that I am presenting, download the one
%% labeled ``slides'' on Indico (or just ignore these yellow pages).
%% \item The annotated version is provided for extra detail and a written
%% record of comments that I intend to make orally.
%% \item Yellow notes refer to the content on the {\it previous} page.
%% \item All other slides are identical for the two versions.
%% \end{notes}

\begin{frame}
\frametitle{The twiki page}

\begin{itemize}\setlength{\itemsep}{0.75 cm}
\item Filled in iCSA08 muon alignment twiki page

\textcolor{blue}{\tt \small \underline{\href{https://twiki.cern.ch/twiki/bin/view/CMS/CSA08AlignMuon}{https://twiki.cern.ch/twiki/bin/view/CMS/CSA08AlignMuon}}}

\vspace{0.25 cm}
\begin{itemize}\setlength{\itemsep}{0.5 cm}
\item Defined iCSA08 workflows

\item Listed event samples, step-by-step procedures, priorities

\item Commented on validation and monitoring
\end{itemize}

\item This talk will be a short summary of the above
\end{itemize}
\end{frame}

\begin{frame}
\frametitle{The workflows}

\begin{itemize}\setlength{\itemsep}{0.1 cm}
\item Whole muon system ``baseline'' alignment
\begin{itemize}
\item align individual chambers relative to the tracker \mbox{(high-$p_T$ tracks)\hspace{-1 cm}}
\item minimal goal (not delivering this would be failure)
\end{itemize}

\item CSC chambers-in-ring alignment
\begin{itemize}
\item two stages: \textcolor{darkblue}{\scriptsize (1)} align CSCs relative to each other in rings and \textcolor{darkblue}{\scriptsize (2)} align rings relative to the tracker
\item step \textcolor{darkblue}{\scriptsize (1)} is the data- and CPU-intensive step
\item special stream to deliver tracks with overlap hits, drawn from normal I.P.\ muons
\item new stream will be tested in 1\_8\_X pre-CSA
\end{itemize}

\item CSC layer alignment
\begin{itemize}
\item align layers relative to layer 1 in each CSC chamber
\item technique is similar to the above, but without overlap requirement (normal AlCa stream)
\end{itemize}

\item CSC beam-halo alignment
\begin{itemize}
\item two parts: chambers-in-ring and layer, like the above
\item technique is the same (or very similar); event \mbox{samples are different\hspace{-1 cm}}
\item new triggers/streams not in pre-CSA; we'll need to test \mbox{in parallel\hspace{-1 cm}}
\end{itemize}
\end{itemize}
\end{frame}

\begin{frame}
\frametitle{Resources Needed}

A little vague at the moment: the baseline procedure is the only one we know well.

\vfill
Baseline requirements:
\begin{itemize}
\item half a million tracks (18~GB), with a filtered copy on T2\_CH\_CAF
\item 50 CPUs for 10 hours
\item 500~MB of AFS space (only 85~MB needs to be saved permanantly)
\end{itemize}

\vfill
{\it Each} of the other workflows will have similar requirements

(including a processed copy of input data on T2\_CH\_CAF)

\vfill
April is devoted to studying each of the other workflows in detail; we
can have quantitative estimates by the end of the month
\end{frame}

\begin{frame}
\frametitle{Monitoring}

\begin{itemize}\setlength{\itemsep}{0.25 cm}
\item Before alignment: put MuonAlignmentAnalyzer in the production jobs?
\item During alignment: 500~MB of diagnostics plots that we can look at in an emergency
\item Just after alignment: 85~MB of summary residuals and database geometry comparisons
\item Just after that: run a small sample through MuonAlignmentAnalyzer
\item In the re-reconstruction: put MuonAlignmentAnalyzer in the production jobs?
\end{itemize}

\vfill
\hspace{-0.83 cm} \textcolor{darkblue}{\Large Output format}

\vfill
\begin{itemize}\setlength{\itemsep}{0.25 cm}
\item Database records in an SQLite file
\item or is direct writing-to-database a goal of this exercise?
\end{itemize}
\end{frame}

\begin{frame}
\frametitle{Conclusions}

\Large

Four workflows considered now, one is essential

\vfill
The others are in development, and will be much better understood by the end of April

\vfill
Much more detail on twiki page

\label{numpages}
\end{frame}


%% \begin{frame}
%% \frametitle{Outline}
%% \begin{itemize}\setlength{\itemsep}{0.75 cm}
%% \item 
%% \end{itemize}
%% %% \hspace{-0.83 cm} \textcolor{darkblue}{\Large Outline2}
%% \end{frame}

%% \section*{First section}
%% \begin{frame}
%% \begin{center}
%% \Huge \textcolor{blue}{First section}
%% \end{center}
%% \end{frame}

%% \begin{frame}
%% \end{frame}

\end{document}
